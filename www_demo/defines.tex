\usepackage[show]{chato-notes}
\usepackage{suffix}
\usepackage{color}
\usepackage{dsfont}
\usepackage{xspace}
\usepackage{amsfonts}
\usepackage{amsmath}
\usepackage{amsthm}
\usepackage{algorithm}
\usepackage[shortlabels]{enumitem}
\usepackage[noend]{algpseudocode}

\newcommand{\comps}{\ensuremath{\boldsymbol{\gamma}}\xspace}
\newcommand{\comp}[1]{\ensuremath{\comps^{#1}}\xspace}
\newcommand{\com}[2]{\ensuremath{\gamma^{#1}_{#2}}\xspace}
\newcommand{\p}[1]{p\left(#1\right)}
\newcommand{\M}{\ensuremath{M}\xspace}

%\newcommand{\comment}[1]{\textcolor{red}{\textbf{[#1]}}}
\newcommand{\squares}{\ensuremath{\mathcal{S}}\xspace}
\newcommand{\neighbors}{\ensuremath{\Delta}\xspace}
\newcommand{\ZLD}{\ensuremath{Z_{LD}}\xspace}
\newcommand{\ZLDlambda}{\ensuremath{Z_{LD}(\lambda)}\xspace}
%\newcommand{\B}{\ensuremath{\mathbf{B}}\xspace}
\newcommand{\B}{\ensuremath{B}\xspace}
\newcommand{\A}{\ensuremath{A}\xspace}
\newcommand{\X}{\ensuremath{\mathcal{X}}\xspace}
\newcommand{\fa}[1]{\quad \forall\,#1}
\newcommand{\wSubL}{\ensuremath{\ell: (i,j,k,\ell) \in \squares}}
\newcommand{\wSubK}{\ensuremath{k: (i,j,k,\ell) \in \squares}}
\newcommand{\xLD}{\ensuremath{\mathbf{\hat{x}}_{\mathrm{LD}}}\xspace}
\newcommand{\yLD}{\ensuremath{\mathbf{\hat{y}}_{\mathrm{LD}}}\xspace}
\newcommand{\xopt}{\ensuremath{x^*}\xspace}
\newcommand{\yopt}{\ensuremath{y^*}\xspace}
\newcommand{\propertyone}{({P1})\xspace}
\newcommand{\propertytwo}{({P2})\xspace}
\newcommand{\natalie}{{\sc Natalie}\xspace}
\newcommand{\prognatalie}{{\sc progNatalie}\xspace}
\newcommand{\prognataliepp}{{\sc progNatalie++}\xspace}
\newcommand{\ld}{{\sc Flan0}\xspace}
\newcommand{\ldf}{{\sc Flan}\xspace}
\newcommand{\meld}{{\sc cFlan}\xspace}
\newcommand{\unary}{{\sc Unary}\xspace}
\newcommand{\cands}{\ensuremath{\mathcal{M}}\xspace}
\newcommand{\nb}{\ensuremath{\bar{n}}\xspace}
\newcommand{\mb}{\ensuremath{\bar{m}}\xspace}
\newcommand{\similarity}{\ensuremath{\sigma}\xspace}

\newcommand{\argmin}{\mathrm{argmin}}
\newcommand{\argmax}{\mathrm{argmax}}
% \atop \wedge k \in V_m}}

%% editing macros
\newcommand{\spara}[1]{\smallskip\noindent{\bf{#1}}}
\newcommand{\mpara}[1]{\medskip\noindent{\bf{#1}}}
\newcommand{\bpara}[1]{\bigskip\noindent{\bf{#1}}}

%% Standard Tatti Stuff
\newcommand{\set}[1]{\left\{#1\right\}}
\newcommand{\pr}[1]{\left(#1\right)}
\newcommand{\fpr}[1]{\mathopen{}\left(#1\right)}
\newcommand{\spr}[1]{\left[#1\right]}
\newcommand{\fspr}[1]{\mathopen{}\left[#1\right]}
\newcommand{\brak}[1]{\left<#1\right>}
\newcommand{\abs}[1]{{\left|#1\right|}}
\newcommand{\norm}[1]{\left\|#1\right\|}
\newcommand{\enset}[2]{\left\{#1 ,\ldots , #2\right\}}
\newcommand{\enpr}[2]{\pr{#1 ,\ldots , #2}}
\newcommand{\enlst}[2]{{#1} ,\ldots , {#2}}
\newcommand{\vect}[1]{\spr{#1}}
\newcommand{\envec}[2]{\vect{#1 ,\ldots , #2}}
\newcommand{\real}{\mathbb{R}}
\newcommand{\np}{\textbf{NP}}
\newcommand{\naturals}{\mathbb{N}}
\newcommand{\funcdef}[3]{{#1}:{#2} \to {#3}}
\newcommand{\define}{\leftarrow}
\newcommand{\reals}{{\mathbb{R}}}

\DeclareRobustCommand{\dispfunc}[2]{%
  \ensuremath{%
  \ifthenelse{\equal{#2}{}}%
    {{#1}}%
    {{#1}\fpr{#2}}}}

\newcommand{\dd}{\ensuremath{K}}

\newcommand{\dist}[1][(p, q)]{\def\ArgI{{#1}}\distRelay}
\newcommand{\distRelay}[2][]{\dispfunc{\dd^{\ArgI}_{#1}}{#2}}
\WithSuffix\newcommand\dist*[2][]{\dist[][#1]{#2}}

\newcommand{\dsc}[1]{\dispfunc{d}{#1}}

%\newcommand{\distsub}[2]{\dispfunc{\dd^{(p, q)}_{#1}}{#2}}
%\WithSuffix\newcommand\distsub*[2]{\dispfunc{\dd_{#1}}{#2}}

%\newcommand{\imp}[1]{\dispfunc{r}{#1}}
\newcommand{\NP}{{\ensuremath{\mathbf{NP}}}}
\newcommand{\coNP}{{\ensuremath{\mathbf{coNP}}}}
\newcommand{\Kendall}{{\ensuremath{K}}}
\newcommand{\wf}{{\ensuremath{w}}}
\newcommand{\wght}[1]{\dispfunc{\wf}{#1}}
\newcommand{\dnst}[1]{\dispfunc{d}{#1}}
\newcommand{\nbhd}[1]{\dispfunc{c}{#1}}
\newcommand{\impin}[1]{\dispfunc{r_{in}}{#1}}
\newcommand{\wghtin}[1]{\dispfunc{w_{in}}{#1}}
\newcommand{\score}[1]{\dispfunc{q}{#1}}
\newcommand{\scorein}[1]{\dispfunc{q_{in}}{#1}}
\newcommand{\fm}[1]{\mathcal{#1}}
\newcommand{\inc}[1]{\dispfunc{in}{#1}}
\newcommand{\median}{\textsc{Median}\xspace}
\newcommand{\greedy}{\textsc{Greedy}\xspace}
\newcommand{\aggrfas}{\textsc{AggrFas}\xspace}
\newcommand{\toydag}{\textsc{ToyDAG}\xspace}
\newcommand{\prob}[1]{p\pr{#1}}
\newcommand{\mean}[2]{\operatorname{E}_{#1}\fspr{#2}}
\newcommand{\ent}[1]{L\fpr{#1}}
\newcommand{\kl}[2]{\mathit{KL}\fpr{{#1} \,\|\, {#2}}}
\newcommand{\pemp}{p^*}
\newcommand{\efrac}[2]{\scriptscriptstyle\frac{#1}{#2}}
\newcommand{\choosetwo}[1]{{\ensuremath{{#1} \choose 2}}}
\newcommand{\degree}[1]{\dispfunc{\mathrm{deg}}{#1}}
\newcommand{\full}[1]{\ensuremath{{#1}_0}}
\newcommand{\fullwght}[1]{\dispfunc{\wf_0}{#1}}

\newcommand{\pedge}{\ensuremath{p_{\mathrm{edge}}}}
\newcommand{\premove}{\ensuremath{p_{\mathrm{remove}}}}
\newcommand{\nswaps}{\ensuremath{N_{\mathrm{swaps}}}}
\newcommand{\padd}{\ensuremath{p_{\mathrm{add}}}}

\newcommand{\wghtmin}[1]{\dispfunc{\wf_m}{#1}}
\newcommand{\wghtsum}[1]{\dispfunc{\wf_s}{#1}}
\newcommand{\wghtnorm}[1]{\dispfunc{\wf_n}{#1}}

%\newtheorem{theorem}{Theorem}
% \newtheorem{lemma}[theorem]{Lemma}
% \newtheorem{proposition}[theorem]{Proposition}
% \newtheorem{corollary}[theorem]{Corollary}
% \newtheorem{definition}[theorem]{Definition}
% %\theoremstyle{definition}
% \newtheorem{example}{Example}
% \newtheorem{problem}{Problem}
% % Hack to get normalfont (amsthm / ntheorem do not work)
% \let\oldexample\example
% \renewcommand\example{\oldexample\normalfont}
% 
% 
% %% Algorithm stuff
% \SetKwComment{tcpas}{\{}{\}}
% \SetCommentSty{textnormal}
% \SetArgSty{textnormal}
% \SetKw{False}{false}
% \SetKw{True}{true}
% \SetKw{Null}{null}
% \SetKwInOut{Output}{output}
% \SetKwInOut{Input}{input}
% \SetKw{AND}{and}
% \SetKw{OR}{or}
% \SetKw{Break}{break}

%% PGF stuff

\iffalse


\pgfdeclarelayer{background}
\pgfdeclarelayer{foreground}
\pgfsetlayers{background,main,foreground}


\definecolor{yafaxiscolor}{rgb}{0.3, 0.3, 0.3}

\definecolor{yafcolor1}{rgb}{0.4, 0.165, 0.553}
\definecolor{yafcolor2}{rgb}{0.949, 0.482, 0.216}
\definecolor{yafcolor3}{rgb}{0.47, 0.549, 0.306}
\definecolor{yafcolor4}{rgb}{0.925, 0.165, 0.224}
\definecolor{yafcolor5}{rgb}{0.141, 0.345, 0.643}
\definecolor{yafcolor6}{rgb}{0.965, 0.933, 0.267}
\definecolor{yafcolor7}{rgb}{0.627, 0.118, 0.165}
\definecolor{yafcolor8}{rgb}{0.878, 0.475, 0.686}

\tikzstyle{exnode} = [inner sep = 1pt]
\tikzstyle{exedge} = [yafcolor5, draw, thick, >=latex, ->]
\tikzstyle{toynode} = [fill = yafcolor5, circle, inner sep = 1.5pt]
\tikzstyle{toyedge} = [draw = black, >=latex, ->, opacity = 0.5]

\tikzstyle{labelnode} = [rectangle, line width = 0pt, text = black, ultra thick, inner sep = 1.5pt, align = center, font = \tiny, minimum height = 0.3cm]

\fi

